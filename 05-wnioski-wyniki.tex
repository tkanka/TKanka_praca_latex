\chapter{Wyniki i wnioski}
%TODO OK to trzeba dołączyć do tego poprzedniego rozdziału.


Aby sprawdzić działanie i dokładność systemu została zaimplementowana możliwość zapisu obliczonego wektora cech na karcie SD. 
Następnie został obliczony przykładowy błąd względny między wektorem cech wyliczonym w implementacji programowej, a uzyskanym z sytemu wizyjnego. %TODO jak programowej, to tu bym się spodziewał czegoś a'la sprzętowej 
Błąd oscyluje w granicy \(10^{-6}\) co czyni go marginalnym i najprawdopodobniej wynika z różnić użytych bibliotek numerycznych.
%TODO To jest niejasne. Rozumiem, że to tylko pokazuje, że ARM i Intel coś inaczej liczną ?

<TU WSTAW WYKRES>

Na przebadanie jednego okna zaproponowany system procesorowy potrzebuje 75ms  (dla porównania te same obliczenia w pakiecie Matlab zajmują około 23 ms).
%TODO podać na jakim sprzęcie 
Dzięki zastosowaniu sprzętowego wyszukiwania ROI zadanie sytemu procesorowego zostało ograniczone do obliczenia jednego okna z największym prawdopodobieństwem zawierania w sobie przechodnia. %TODO styl. to zawierania w sobie, poza tym - dalczego akurat tylko jednego.
Kamera termowizyjna, będąca źródłem sygnału dla wzorca probabilistycznego, pracuje z prędkością 9 klatek na sekundę dając w przybliżeniu 111 ms na zbadanie danego okna więc system procesorowy mieści się w tych ramach czasowych z dużym zapasem. %TODO dając - słabe słowo
%TODO  no ale jakby to dla 30/60 fps... to już tak pięknie nie jest

\begin{table}[]
\centering
\caption{Wykorzystane zasoby logiki programowalnej.}
\label{tab:resources}
\begin{tabular}{|l|l|l|l|}
\hline
Resource & Utilization & Available & Utilization \% \\ \hline %TODO po polsku
LUT & 12583 & 17600 & 71,49 \\ \hline 
LUTRAM & 617 & 6000 & 10,28 \\ \hline 
FF & 19924 & 35200 & 56,60 \\ \hline
BRAM & 25,50 & 60 & 42,50 \\ \hline
DSP & 36 & 80 & 45,00 \\ \hline
IO & 43 & 100 & 43,00 \\ \hline
BUFG & 7 & 32 & 21,88 \\ \hline
MMCM & 1 & 2 & 50,00 \\ \hline
PLL & 1 & 2 & 50,00 \\ \hline
\end{tabular}
\end{table}


%TODO Podsumowanie oraz wskazanie dalszych kierunków pracy
%TODO Dodatek A - spis zawartości CD
%TODO Dodatek B - opis informatyczny projektu