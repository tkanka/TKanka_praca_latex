\chapter{Wykorzystanie FPGA w analizie obrazu}
\label{cha:fpga}
Tradycyjne systemy wizyjne zwykle bazują na architekturze sekwencyjnej.
W tym rozwiązaniu obraz jest sukcesywnie poddawany kolejnym przekształceniom, a~wyniki pośrednie zapisywane są w pamięci operacyjnej. 
W aplikacji procesorowej operacje te są wykonywane przez układ arytmetyczno-logiczny. 
Kolejne kroki algorytmu są kompilowane w ciąg instrukcji dla procesora ,który oprócz operacji matematycznych dużą część pracy poświęca na pobieranie i dekodowanie rozkazów oraz na odczytywanie i zapisywanie danych do pamięci. 
By taka aplikacja mogła pracować w czasie rzeczywistym, cała procedura musi wykonać się szybciej niż przychodzące dane obrazu, co wymusza wysokie taktowanie procesora sięgające kilku GHz. To podejście jednak ma swoje ograniczenia. Wraz z wzrostem częstotliwości pracy procesora wzrasta jego moc a tym samym ilość ciepła które która musi zostać rozproszone. Najszybsze CPU są taktowane z częstotliwością zegara sięgającą nawet 4,4 GHz (choć przy zastosowaniu chłodzenia ciekłym azotem jest możliwe uzyskanie ponad 8 GHz). Wzrost szybkości obliczeń  uzyskuje się coraz częsciej poprzez zwiększanie ilości rdzen wi procesorze. %TODO dodać, że to i tak nie zawsze pomaga. Poza tym wypada jakoś skomentować fakt istnienie procesorów wielordzeniowych.

W przypadku podejścia równoległego, implementacja poszczególnych kroków algorytmu odbywa się w osobnych procesach. Jeżeli wykonywany algorytm jest głównie sekwencyjny tzn. kolejne kroki algorytmu wymagałyby danych otrzymanych z poprzednich, to zysk takiego zabiegu byłby równy zero. W celu uzyskania dobrej implementacji w układzie równoległym, istotne jest by znaczna cześć algorytmu mogła być wykonywana równolegle. 
%TODO OK to jest dość niejednoznaczne. Musi Pan to opatrzyć jakimś lepszym komentarzem/przykładem.
Maksymalne do uzyskania przyspieszenie jest określone przez prawo Amdahla: 
\begin{equation}
P_w =\frac{1}{ s + \frac{1-s}{n_w}}
\end{equation}
gdzie:
\begin{conditions}
P_{w} & przyspieszenie algorytmu w systemie wieloprocesorowym, \\
s & cześć algorytmu niepodlegająca zrównolegleniu (wartość od zera do jeden), \\
n_{w} & liczba elementów obliczeniowych.
\end{conditions}

Algorytmy przetwarzania obrazów są w dużej mierze równoległe, szczególnie te niskiego i średniego poziomu. W wielu przypadkach każdy piksel obrazu można obliczyć niezależnie, np. w wszelkich operacjach kontekstowych, przekształceniach przestrzeni barw, binaryzacji itp. Między innymi dlatego układy FPGA są chętnie stosowane w systemach wizyjnych. Teoretycznie jedynym ograniczaniem w możliwości zrównoleglenia obliczeń jest liczba dostępnych zasobów w układzie, jednak innym istotnym aspektem jest sposób dostarczania danych do modułów obliczeniowych. 
%TODO procesorów, czy modułów obliczeniowych no bo w sumie to musi się Pan ogólnie zdecydować, czy pisze to Pan ogólnie, czy z uwzględnieniem architektury.

Dostęp do pamięci często wymaga czasu a ilość danych przekazana podczas jednego transferu jest ograniczona. Stanowi to wąskie gardło w tego rodzaju rozwiązaniach.

% USUNIĘTE Czas i przepustowość jaka jest potrzebna do odczytania i zapisu obrazu po przetworzeniu pomiędzy pamięci jest najczęściej wąskim gardłem systemu wizyjnego. %TODO OK to z i do źle wygląda
Z tego powodu przetwarzanie obrazu bezpośrednio z sensora w czasie jego akwizycji jest chętnie wykorzystywane, gdyż zmniejsza to liczbę operacji odczytu i zapisu. \cite{garcia2014survey}

%TODO OK No i w ten sposób to Pan w rodziale o FPGA, ani razu nie użył słowa FPGA.
%TODO Druga sprawa to jest GPU - tez się tu powinno pojawić.


%TODO OK To trzeba wkomponować w rozdział o algorytmach. Ew. rozdział FPGA można dać przed algorytmy i wtedy omówić metody programowe oraz sprzętowe.

\section{Zynq-7000}

Rodzina układów Zynq-7000 bazuje na architekturze SoC (ang. \textit{System on Chip}). W pojedynczym układzie scalonym został zawarty kompletny system w skład którego wchodzą układy spełniające różne funkcje. Został on podzielony na dwie główne części: systemu procesorowy (PS ang. \textit{Porcessing System}) bazujący na procesorze ARM Cortex-A9 oraz logikę programowalną (PL ang. \textit{Programable Logic}) - FPGA. %TODO OK potworek styl. -> poprawić. + nazwy/słowa angielskie zawsze w kursywie
Na rysunku \ref{fig:zynq7000} przedstawiono schemat architektury układu. %TODO OK czego ?
Część procesorowa, oprócz samego ARM-a, posiada wbudowaną pamięć, kontroler pamięci zewnętrznej oraz szereg interfejsów dla układów peryferyjnych takich jak USB, GigEthernet, CAN, I2C, SPI. %TODO OK piwersza część do poprawy (styl.)
W części logiki programowalnej znajdują się bloki logiki konfigurowalnej (CLB ang. \textit{configurable logic block}), 36Kb bloki pamięci RAM, moduły DSP48, układ JTAG, układy zarządzania zegarami oraz dwa 12-bitowe przetworniki analogowo-cyfrowe.
%TODO 1. nie nazywałbym tego procesorem sygnałowym. 2. jednak bardziej obszerny opis obu komponentów (tak na 1/2 strony każdy)

Komunikacja między częścią procesorową, a logiką programowalną odbywa się za pośrednictwem interfejsu AXI (ang. \textit{Advanced Extensible Interface}) oraz bezpośrednio wykorzystując porty ogólnego przeznaczenia, przerwania i bezpośredni dostęp do pamięci (DMA ang. \textit{Direct Memory Access}).
%TODO 1. powt. oraz, 2 DMA, z tego co wiem, jest via AXI. Może chodziło Panu o ACP ?

\begin{figure}[h]
    \centering
    \includegraphics[width=1\textwidth]{images/Zynq-7000-Overview}
    \caption{Schemat ogólny architektury układu Zynq-7000.}
    \label{fig:zynq7000}
\end{figure}

\section{Interfejs AXI}
 AXI (ang. \textit{Advanced eXtensible Interface} --  zaawansowany rozszerzalny interfejs) jest częścią ARM AMBA (ang.\textit{ Advanced Microcontroller Bus Architecture}) -- otwartego standardu, będącego specyfikacją do zarządzania połączeniami między blokami funkcyjnymi w SoC. 
%TODO OK gdzie kończy się nawias ?
 Aktualnie jest stosowana AMBA 4.0 która wprowadziła drugą wersję AXI -- AXI4. 
 Występują trzy typy interfejsów dla AXI4:
\begin{itemize}
\item AXI4 -- stosowany w wysokowydajnych transferach w przestrzeni pamięci (ang. \textit{memory-mapped}),
\item AXI4-Lite -- stosowany dla prostszych operacji w przestrzeni pamięci (na przykład do komunikacji z rejestrami kontrolnymi i statusu),
\item AXI4-Stream --stosowany do transmisji strumieniowych (wysokiej prędkości). 
\end{itemize}
Specyfikacja interfejsu zakłada komunikację pomiędzy pojedynczym AXI \textit{master} i pojedynczym AXI \textit{slave}, która ma na celu wymianę informacji. 
Kilkanaście interfejsów AXI \textit{master} i \textit{slave} mogą zostać połączone między sobą za pomocą specjalnej struktury zwanej \textit{interconnect block} (blok międzypołączeniowy), w której odbywa się trasowanie połączeń do poszczególnych bloków. 

AXI4 i AXI4-Lite składają się z 5 różnych kanałów:
\begin{itemize}
\item Kanał adresu odczytu,
\item Kanał adresu zapisu,
\item Kanał danych odczytanych
\item Kanał danych do zapisania
\item Kanał potwierdzenia zapisu
\end{itemize}
Dane mogą płynąć w obie strony pomiędzy \textit{master} a \textit{slave} jednocześnie.
Ilość danych, które można przesłać w jednej transakcji w przypadku AXI4 wynosi 256 transferów, zaś AXI4-Lite pozwala na tylko 1 transmisję.

%TODO te słowa master i slave to kursywą.


AXI4-Stream nie posiada pola adresowego, a dane mogą być przesyłane nieprzerwanie. 