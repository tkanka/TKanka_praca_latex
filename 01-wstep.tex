\chapter{Wstęp}
\label{cha:wstep}

Cyfrowa analiza obrazów znalazła szerokie zastosowanie w wielu dziedzinach życia. 
Umożliwia ona automatyczne uzyskanie istotnych dla rozważanego systemu informacji na podstawie obrazu bez konieczności angażowania człowieka.
%TODO jakiś przykład ?
Niektóre informacje zawarte w obrazie nie są dobrze dostrzegane przez ludzką percepcję np. kolor jest bardzo subiektywnym parametrem dla różnych ludzi. %TODO pomijając styl, to nie za bardzo rozumiem po co to zdanie - bardziej w nawiązaniu do Pana pracy byłoby pójście w kierunku kamer IR, termo, że percepcja u ludzi jest ograniczona itp.
Przez ostanie kilkadziesiąt lat opracowano tysiące różnych technik i algorytmów wyspecjalizowanych do określonych zadań np. kontrola jakości i przebiegu procesu przemysłowego, kontrola dostępu poprzez rozpoznawanie twarzy w smartfonach iPhone, optymalizacja ruchu na skrzyżowaniach, bezobsługowe systemy bezpieczeństwa i monitoringu, pojazdy autonomiczne, leśne fotopułapki do badania zachowań i migracji zwierząt itp. 
%TODO tu bym zastosował "gradację" od prostych do złożonych - a fotopułapka to chyba najprostsza jest.
Dzisiejsza technologia nie ogranicza nas tylko do stosowania spektrum światła widzialnego ludzkim okiem. 
Kamery na podczerwień stają się coraz tańsze i coraz bardziej popularne.
%TODO Proszę uważać, bo IR/na pod czerwień to co innego, a termowizyjne to co innego. Te pierwsze to wizyjne z zwiększoną czułością w paśnie podczerwonym. Ale może Pan o nich też wspomnieć.
Dostarczają nam informacje o temperaturze obserwowanych obiektów i jest coraz chętniej wykorzystywane w wielu różnych dziedzinach np. weterynarii do określenia miejsc urazów zwierząt, kontroli jakości artykułów spożywczych, analiza strat cieplnych w budynkach, detekcji gazów, systemów wspomagania kierowcy\cite{gade2014thermal}. 
%TODO dron, poszukiwania, policja, straż graniczna, gruzowsika..

Większość systemów wizyjnych służących do rozpoznawania przechodniów jest oparte o analizę obrazów z zakresu światła widzialnego, bądź podczerwieni. 
W przypadku światła widzialnego można uzyskać bardzo dobre wyniki pod warunkiem że wyszukiwane obiekty są dobrze oświetlone i wyróżniają się swoim kolorem od tła. 
Podczerwień, a szczególnie termowizja, umożliwia detekcję w warunkach nocnych i ograniczonej widoczności. %TODO tu też ostrożnie
Oba podejścia mają swoje wady i zalety, które wzajemnie się uzupełniają np. duże nasłonecznienia powoduje, że tło termiczne staje się dużo wyższe co utrudnia wyodrębnienie pieszego, natomiast daje idealne warunki do uzyskania dobrej jakości obrazu w zakresie widzialnym \cite{lee2015robust}. 
Połączenie tych dwóch obrazów daje możliwość uzyskania jeszcze lepszych metod rozpoznawania ludzi. %TODO nie metod, tylko skuteczności.
W pracy \cite{st2007combination} autorzy nazywają ten rozszerzony format jako RGBT (''Red-Green-Blue-Thermal''), natomiast inna praca jako analizę wielospektralną (Multispectral) \cite{hwang2015multispectral}, albo po prostu jako połączony obraz z kamery termowizyjnej i zwykłej\cite{lee2015robust}. 

Skuteczna detekcja obiektów jest często okupiona dużym zapotrzebowaniem na zasoby obliczeniowe. 
W wielu przypadkach nie da się uzyskać satysfakcjonującej wydajności by można było uznać system za działający w czasie rzeczywistym wykorzystując jedynie typowy komputer wyposażony w~procesor ogólnego przeznaczenia.
Daję to pole do popisu dla układów rekonfigurowalnych które mają możliwość dużego zrównolegnienia obliczeń. %TODO styl. "pole do popisu". Poza tym, by to bardziej ogólnie podszedł, że anceleracji i zahaczył o ASIC, GPU i FPGA, czy Zynq
Układy FPGA (ang. \textit{Field-Programmable Gate Array}) znalazły już zastosowanie w wielu systemach wizyjnych wykonując różnego rodzaju niskopoziomowe operacje kontekstowe, zamiany przestrzeni barw czy też binaryzacji nawet w czasie jednego cyklu zegara. 
%TODO No tu by się Pan bardziej postarał, bo to co Pan wymienił to początniem lat 90 robiono.
Dodatkową zaletą układów FPGA jest mały pobór mocy, co czyni je niezwykle atrakcyjne dla  aplikacji mobilnych -- takich jak drony czy czujniki środowiskowe \cite{garcia2014survey}. 

Niniejsza praca stanowi kontynuację i rozwinięcie pracy inżynierskiej autora.

\section{Cel pracy}


Celem pracy była realizacja wbudowanego systemu wizyjnego do detekcji wybranych obiektów (np. ludzi) na podstawie obrazu z kamery termowizyjnej. %TODO Tylko termo ?
Zakłada się, że jako platforma obliczeniowa zostanie użyty układ heterogeniczny (np. Zynq firmy Xilinx), który umożliwia realizację sprzętowo-programową algorytmów.

\section{Struktura pracy}

%TODO Tutaj standardowe "w rozdziele" + \ref{.}

W pierwszej części została opisana budowa cyfrowego systemu wizyjnego z wykorzystaniem połączonych obrazów RGB oraz IR. 
Zawiera on [rozdział] teorię stanowiącą podstawę dla realizowanych prac oraz kilka przykładów już zrealizowanych systemów. 
W następnym rozdziale została podana specyfikacja techniczna zastosowanych urządzeń oraz technologii. %TODO specyfikacja techniczna technologii dziwnie brzmi
W rozdziale czwartym opisano realizację autorskiego systemu detekcji ludzi. 
Prace zakończono omówieniem uzyskanych wyników, wnioskami oraz wskazaniem dalszych kierunków rozwoju stworzonego systemu.