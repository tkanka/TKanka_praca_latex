\chapter{Wprowadzenie}
Cyfrowa analiza obrazów znalazła szerokie zastosowanie w wielu dziedzinach życia. Dzięki niej możliwe jest automatyczne uzyskanie istotnych dla użytkownika informacji. Przez ostanie kilkadziesiąt lat opracowano tysiące różnych technik i algorytmów wyspecjalizowanych do określonych zadań np. leśne fotopułapki do badania zachowań i migracji zwierząt, kontrola jakości i przebiegu procesu przemysłowego, kontrola dostępu poprzez rozpoznawanie twarzy m.in. w smartfonach, analiza zdjęć satelitarnych ziemi umożliwia prognozowanie pogody, obraz z kamer zamontowanych nad skrzyżowaniami usprawnia przejazd pojazdów a poprzez badanie przekroju żołędzia można określić czy jest dobrym kandydatem na sadzonkę. 

Wzrok ludzki operuje w pewnym zakresie promieniowania elektromagnetycznego zwanego światłem widzialnym. Dzisiejsza technologia daje możliwość rejestracji obrazów wykraczającym poza to widmo. Kamery termowizyjne stają się coraz tańsze i coraz bardziej popularne. Dostarczają nam informacje o temperaturze obserwowanych obiektów. Jest to coraz chętniej wykorzystywane np. w weterynarii do określenia miejsc urazów zwierząt, w przemyśle do kontroli jakości artykułów spożywczych, w budownictwie do analizi strat cieplnych w budynkach, w sytemahc wspomagania kierowcy, przez ratowników do odnajdywania zasypanych ludzi w gruzowiskach, straż graniczną do monitorowania granic, przez wojsko do odnajdywania celów i zagrożeń podczas misji m.in. z wykorzystaniem do tego celu dronów. \cite{gade2014thermal}. 

Większość systemów wizyjnych służących do rozpoznawania przechodniów jest oparte o analizę obrazów z zakresu światła widzialnego, bądź podczerwieni. 
W przypadku światła widzialnego można uzyskać bardzo dobre wyniki pod warunkiem że wyszukiwane obiekty są dobrze oświetlone i wyróżniają się swoim kolorem od tła. 
Podczerwień, a szczególnie termowizja, umożliwia detekcję w warunkach nocnych i ograniczonej widoczności.
Oba podejścia mają swoje wady i zalety, które wzajemnie się uzupełniają np. duże nasłonecznienia powoduje, że tło termiczne staje się dużo wyższe co utrudnia wyodrębnienie pieszego, natomiast daje idealne warunki do uzyskania dobrej jakości obrazu w zakresie widzialnym \cite{lee2015robust}. 
Połączenie tych dwóch obrazów daje możliwość uzyskania jeszcze lepszej skuteczności rozpoznawania ludzi. W pracy \cite{st2007combination} autorzy nazywają ten rozszerzony format jako RGBT (''Red-Green-Blue-Thermal''), natomiast inna praca jako analizę wielospektralną (Multispectral) \cite{hwang2015multispectral}, albo po prostu jako połączony obraz z kamery termowizyjnej i zwykłej\cite{lee2015robust}. 

Skuteczna detekcja obiektów jest często okupiona dużym zapotrzebowaniem na zasoby obliczeniowe. W wielu przypadkach nie da się uzyskać satysfakcjonującej wydajności by można było uznać system za działający w czasie rzeczywistym wykorzystując jedynie typowy komputer wyposażony w~procesor ogólnego przeznaczenia. Wykorzystuje się różne metody akceleracji tych obliczeń. Karty graficzne pozwalają na duże zrównoleglenie obliczeń okupione znacznym zużyciem energii. Tworzenie specjalizowanych układów scalonych (ASIC) daje najlepsze rezultaty w implementacji systemu wizyjnego ale ich opracowanie i produkcja wymaga bardzo dużych nakładów finansowych. Dobrym rozwiązaniem stanowią układy rekonfigurowalne które dają podobne możliwości w realizacji wyspecjalizowanych zadań co układy ASIC ale nie wymagają dużych nakładów finansowych w ich tworzeniu.

Układy FPGA (ang. \textit{Field-Programmable Gate Array}) mają ogromną możliwość do zrównoleglenia obliczeń i są chętnie stosowane w systemach wizyjnych. Szczególnie chętnie są wykorzystywane do operacjach niskiego poziomu, przygotowując wstępnie obraz do dalszej analizy na wysokim poziomie takie jak: filtry konwolucyjne, filtry 2D, subsamplowanie, wykrywanie krawędzi, obliczanie SAD z regionu zainteresowania, obliczanie oriętacji krawędzi i histogramów, obliczanie strumieniowo statystyk (wartość maksymalna, minimalna, średnia), zmiana przestrzeni barw./cite{ kisacanin2008embedded}
Dodatkową zaletą układów FPGA jest mały pobór mocy, co czyni je niezwykle atrakcyjne dla aplikacji mobilnych -- takich jak drony czy czujniki środowiskowe \cite{garcia2014survey}. 
Układy heterogeniczne łączą w jednej obudowie dwa układy o różanej architekturze i funkcjonalności. Przykładem takiego połączenia jest Zynq-7000 firmy Xilinx który łączy w sobie układ FPGA oraz procesor ARM. Największą zaletą takiego połączenia jest wysoka przepustowość danych między procesorem a logiką programowalną. 

Niniejsza praca stanowi kontynuację i rozwinięcie pracy inżynierskiej autora.

\section{Cel pracy}


Celem pracy była realizacja wbudowanego systemu wizyjnego do detekcji wybranych obiektów (np. ludzi) na podstawie obrazu z kamery termowizyjnej oraz konwencjonalnej. 
Zakłada się, że jako platforma obliczeniowa zostanie użyty układ heterogeniczny (np. Zynq firmy Xilinx), który umożliwia realizację sprzętowo-programową algorytmów.

\section{Struktura pracy}

W pierwszej części została opisana budowa cyfrowego systemu wizyjnego z wykorzystaniem połączonych obrazów RGB oraz IR. 
Rozdział \ref{cha:csw}zawiera teorię stanowiącą podstawę dla realizowanych prac oraz kilka 
przykładów już zrealizowanych systemów. 
W rozdziale \ref{cha:hw} zostały przedstawione wykorzystane zasoby sprzętowe oraz technologie użyte w opracowaniu systemu wizyjnego.
W rozdziale \ref{cha:real} zawiera realizację autorskiego systemu detekcji ludzi. 
Prace zakończono omówieniem uzyskanych wyników, wnioskami oraz wskazaniem dalszych kierunków rozwoju stworzonego systemu.
