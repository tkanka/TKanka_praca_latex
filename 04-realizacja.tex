\chapter{Realizacja}
\label{cha:real}
%TODO Zły tytuł !!! mało informatywyny

%TODO Na początku tego rozdziału koncepcja systemu.


W celu detekcji pieszych, wykorzystany został połączony obraz termowizyjny (IR) i kolorowy (RGB) nazywany dalej RGBIR. 
Następnie ten obraz zostaje poddany analizie HOG oraz klasyfikacji za pomocą SVM. 
W celu ustalenia obszaru zainteresowania na obrazie termowizyjnym za pomocą wzorca probabilistycznego zostają wytypowani kandydaci. %TODO styl.

%TODO To jest jakaś koncepcja, ale trzeba to rozwinąć i dodać schemat. Także opisać co, jak i gdzie i dlaczego zosało zaimplementowane (FPGA/ARM)

\section{Akwizycja obrazu}

Obraz kolorowy służy jako obraz bazowy. %TODO raczej z kamery wizyjnej
Rozdzielczości 640 x 480 pikseli, prędkością 30 klatek na sekundę i głębi 8 bitów na kanał. 
Źródłem tego obrazu jest kamera podłączona do układu za pomocą interfejsu HDMI. 
%TODO Te 3 zdania przeredagować i poprawić.

Na obraz bazowy zostaje nałożony obraz termowizyjny z kamery Lepton, który różni się znacząco parametrami. %TODO podać szczegóły tego zróżnicowania.

Abu je zsynchronizować zastosowano bufor ramki, do którego jest zapisywany obraz z prędkością 9 klatek na sekundę, a odczytywany z prędkością 30. 
Kolejnym przekształceniem jest transformacja projekcyjna. 
Ma ona na celu powiększenie i dopasowanie obrazu termowizyjnego, tak by poprawnie pokrywał się z obraz wizyjnym. 
W tym celu został zaimplementowany moduł, który oblicza na podstawie parametrów macierzy transformaty i współrzędnych piksela obrazu źródłowego odpowiadającą mu pozycję na obrazie termowizyjnym zapisanym w buforze ramki. %TODO To jest ciut niejasne.
 
Następny moduł dokonuje interpolacji dwuliniowej. 
Do poprawnej interpolacji wymagane są 4 piksele otaczające obliczony z projekcji punkt. 
W celu zredukowania liczby dostępów do pamięci i zwiększenie szybkość działania, moduł zapamiętuje 4 ostatnio użyte wartości pikseli. 
%TODO więcej szczegółów
Rozwiązanie to pozwala na pracę w czasie rzeczywistym małym kosztem zasobów układu.

Strumień wizyjny jak i termowizyjny działają w AXI-Stream. %TODO raczej przekazywane/przesyłane są
Umożliwia to łatwą synchronizację obu obrazów na podstawie sygnału SOF (ang. \textit{Start o frame}). 
Moduł synchronizacji oczekuje na pojawienie się tego sygnału w strumieniu termowizyjnym. 
Do tego momentu wszystkie napływające piksele są odrzucane. 
Gdy pojawi się sygnał, strumień IR zostaje zatrzymany i czeka na pojawienie się sygnał SOF w bazowym strumieniu wizyjnym. 
%TODO No własnie mógłby Pan coś opisać o komunikacji z tą kamerą (widzę, że jest później) - zaproponuje jednak innyt układ treści.
Po jego wykryciu strumień IR rusza. %TODO styl.
Oba strumienie zostają zsynchronizowane tworząc strumień wizyjny obrazu RGBIR. 
Następnie ten strumień zostaje przesłany do pamięci za pośrednictwem VDMA oraz (po koloryzacji i nałożeniu) wyświetlony na monitorze przez port VGA.
%TODO za dużo razy strumień.

\section{Kalibracja}

Aby obraz termowizyjny poprawnie pokrywał się z obrazem RGB należy wykonać procedurę kalibracji. 
Kalibracja przeprowadzana jest ręcznie. 
Oprogramowanie kamery pozwala na zapisanie na karcie SD specjalnego obrazu kalibracyjnego, na którym jest zawarty zrzut aktualnie wyświetlanego obrazu wraz z nieprzetworzonym projekcyjnie obrazem IR. %TODO ale to kamery, czy to co Pan zrobił ? (tzn. zybo) i co to jest aktualnie wyświetlny obraz. Chyba chodiz o to, że obrazu z kamery wizyjnej i termowizyjnej...
Następnie w pakiecie Matlab zostaje obliczona macierz transformaty projekcyjnej za pomocą wbudowanej funkcji. 
Wymaga ona wskazania 4 par odpowiadających sobie punktów na obrazie RGB oraz IR. 
Nową macierz można wgrać podając jej parametry w konsoli. 

%TODO tutaj jakis przykład (w sensie zdjęcie)


\section{Wyznaczanie ROI}

Strumień IR z kamery zostaje zbinaryzowany i poddany analizie w detektorze DPM korzystającym z wzorca probabilistycznego.  %TODO odnośnik do rozdziału z opism teorii.
Moduł DPM przesyła do pamięci listę koordynatów kandydatów wraz z mocą dopasowania. %TODO mocą dopasowania ?
Moduł DPM został zaczerpnięty z pracy inżynierskiej. %TODO powt. moduł DPM
Moduł wykorzystuję strumień bezpośrednio z kamery. %TODO powt. moduł...
Wielkość okna detekcji wynosi 16 x 40 pikseli. 
Jeżeli badany obraz binarny wykazał odpowiedni poziom dopasowania do wzorca, zostaje wysłana o tym informacja poprzez AXI-Stream do pamięci. 
%TODO stream ? jakoś tego nie rozumiem.
Zawiera ona koordynaty okna w układzie odniesienia kamery IR oraz wartość mocy dopasowania. 
Gdy zostanie zbadane ostatnie okno w obrazie, zostaje wysłany sygnał TLAST co wygeneruje przerwanie dla systemu procesorowego.

%TODO To też  jest bardzo skrótowo opisane. Nie musi tu Pan całej inżynierki przepisywać, ale ... no ciut wiecej szzcegółów (tak by było to kompletne)

%TODO Na pracę inż. proszę się powołać jak na każdą inną publikację...

\section{Klasyfikacja za pomocą HOG+SVM}

Z lisy kandydatów wygenerowanej przez moduł DPM wybierany jest wynik o najwyższej mocy dopasowania. %TODO dlaczego tylko jeden ?
Koordynaty z układu odniesienia kamery zostają poddane transformacji projekcyjnej do układu odniesienia kamery RGB. 
Z obszaru na obrazie RGBIR zawierającym potencjalnie człowieka zostają wyodrębnione cechy HOG, które następnie służą jako wektor dla SVM.

Klasyfikator został opracowany i nauczony na podstawie 60 wyselekcjonowanych obrazów. %TODO trochę mało - nie miał Pan wiecej próbek ?
30 z nich stanowiło próbką pozytywną zawierającą osobę, a 30 negatywną. 
Nauczanie zostało zrealizowane przy użyciu oprogramowania Matlab. 
Próbki pozytywne zostały wygenerowane poprzez zapis ROI wyznaczonych przez wzorzec probabilistyczny. 

%TODO nie można pozyskać próbek z jakieś bazy ? 
%TODO parametry tego HOG'a SVM'a. Rozumiem, że jest on w PS uruchomiony - też to nie jest napisane.

\section{Prezentacja wyników}

Na wyjściu konsoli zostają podane współrzędne oraz moc dopasowania i klasyfikacja obiektu. %TODO zbyt skórtowe i niejasne
Na obrazie wyjściowym VGA obszar ten zostaje zaznaczony zieloną ramką. 
Jeżeli potencjalny obszar nie został zakwalifikowany jako człowiek, ale miał największą moc dopasowania DPM to obszar zostaję zaznaczony czerwoną ramką. Czarna ramka oznacza, że nie został wykryty żaden obiekt. %TODO tego nie rozumiem.

%TODO To zdanie wcześniej /poczatek rozdziału/
Mając do dyspozycji układ heterogeniczny z~rodziny Zynq-7000 firmy Xilinx, operacje zostały podzielone między logikę programowalną, a system procesorowy. Ogólny zarys rozwiązania został przedstawiony na rysunku \ref{fig:systemwizyjny}.

\begin{figure}[h]
    \centering
    \includegraphics[width=1\textwidth]{images/system}
    \caption{Schemat blokowy systemu detekcji.}
    \label{fig:systemwizyjny}
\end{figure}

Logika programowalna :
\begin{itemize}
\item Akwizycja obrazu poprzez HDMI (RGB) i VoSPI (IR),
\item Transformata projekcyjna i interpolacja obrazu IR,
\item Nałożenie i synchronizacja obrazu IR do obrazu RGB,
\item Prezentacja wyników,
\item Detekcja kandydatów za pomocą wzorca probabilistycznego.
\end{itemize}
System procesorowy:
\begin{itemize}
\item konfiguracja parametrów systemu wizyjnego w logice programowalnej poprzez interfejs AXI-Lite,
\item Klasyfikacja obszarów wytypowanych przez wzorzec probabilistyczny,
\item Generowanie oznaczników. %TODO co to są oznaczniki
\end{itemize}

\section{Opis modułów}

%TODO Tu podam moją koncepcję tego rodziału. Na początku opis elementów, schemat ogólny (mniej szczegółowy, niż ten co jest) + podział na HW/SW z jakimś uzasadnieniem. Potem integrowane omówienie z opisem szczegółowym (czyli to co były wyżej, z tym co jest tutaj.)

\subsection{Kontroler kamery IR}

Kontroler odpowiada za pobieranie obrazu z kamery IR poprzez interfejs VoSPI, który następnie zostaje zapisany do dwuportowej pamięci BRAM. 
Na początku pracy w stan niski ustawiany jest pin CS (ang. \textit{Chip Select}), a po chwili rozpoczyna transmisję poprzez taktowanie zegarem SCK. %TODO pin czego ? kamery kto rozpoczyna transmisję.
Kamera reaguje na opadające zbocze zegara i wystawia kolejny bit danych na swoim porcie MISO. %TODO skrót ?
Strumień VoSPI składa się z 63 pakietów na ramkę obrazu. 
Pakiet rozpoczyna identyfikator składający się z numeru linii oraz sumy CRC pakietu (2 bajty na numer linii i 2 na sumę). 
Dane pakietu stanowi 160 bajtów -- po dwa bajty na piksel w linii. 
Dane są przesyłane w 14-bitową wartość piksela oraz 2 zera wypełnienia. %TODO coś tu nie gra styl.
W przypadku niepoprawnej ramki numer identyfikator przyjmuje wartość xFxx. 
Ostatnie trzy pakiety stanowią telemetrię i są ignorowane. %TODO ale co tu jest rozumiane przez telemetrię.

%TODO Ogólnie to proszę rozwinąć ten opis. Omówić ten moduł co Pan zrobił, jak to jest dekodowane itp. Jak komunikacja z czujnikiem.

\subsection{Transformata projekcyjna}

Moduł zamienia współrzędne z układu odniesienia kamery RGB odpowiadającym im na obrazie IR. %TODO styl.
Na wejściu podawany jest strumień AXI4-Stream zawierający timingi oraz 12 bitowe współrzędne X i Y. %TODO timinig (zmienić słowo)
Moduł realizuję operację: 

\begin{equation}
\begin{bmatrix}
u_n & v_n & n
\end{bmatrix} 
= 
\begin{bmatrix}
x & y & 1
\end{bmatrix}
T
\end{equation}

\begin{equation}
u = \frac{u_n}{n}
\end{equation}

\begin{equation}
v = \frac{v_n}{n}
\end{equation}

%TODO opisać te symbiole

Moduł wystawia na wyjściu strumień timingów, 12 bitowe wartości U i V oraz ich części ułamkowe w U\_fraction i V\_fraction (14 bitów). %TODO tez to timinigów.
W module zostały wykorzystane 34 z 80 dostępnych w układzie Zynq procesorów DSP48 do wykonania operacji arytmetycznych. %TODO modułów DSP48
Najwięcej zasobów jest pochłonięte przez moduł dzielarki dostarczony od producenta układu. %TODO styl.
Do implementacji jednej dzielarki zostało wykorzystane 14 modułów DSP. %TODO powt. to jakoś trzeba złączyć z poprzednim 
Dzielenie nie odbywa się w pełni potokowo. 
Użyty w dzielarce algorytm High\_Radix wymaga zatrzymania strumienia na czas obliczeń. 
Jednak dzięki zastosowaniu wyższej częstotliwości niż zegar pikseli obrazu RGB oraz bufora (250 MHz) nie stanowi to wąskiego gardła systemu. %TODO a w innym trybie to nie działa w pełni potokow ? Swoją drogą, czy nie można jakoś tego uniknąć ?
Macierz T jest zapisana  w dziewięciu 32 bitowych rejestrach i konfigurowalna poprzez interfejs AXI4-Lite. 
Elementy macierzy są 25 liczbami w notacji stałoprzecinkowej: 1 bit znaku 10 – część całkowita, 14 – część ułamkowa.

%TODO Też jakiś schemat obliczeń i więcej szczegółów.



\subsection{Interpolacja bilinearna} %TODO może lepiej dwuliniowa

Prosty moduł przeznaczony głównie do powiększania obrazów. %TODO ??? styl. co to ma być
Ma za zadanie pobrać z pamięci dwuportowej obrazu IR wartość piksela wskazaną na wejściu układu i wystawić na wyjście. %TODO ??? a coś jeszzce zrobić - bo to jest kontorler....
Podobnie jak reszta systemu używa AXI4-Stream do przekazywania danych między poszczególnymi modułami. 
Dane na wejściu to współrzędne U i V oraz ich części ułamkowe U\_fraction i V\_fraction. 
Moduł został wyposażony w 4 rejestry, w których przechowywane są współrzędne oraz wartości 4 ostatnio użytych pikseli. 
Zabieg ten znacznie redukuje liczbę potrzebnych zapytań do pamięci. 
Podczas powiększania obrazów jest duża szansa, że kolejne koordynaty na wejściu UV odwołują się do tych samych czterech otaczających ich pikseli. 
%TODO napisac dlaczego tak jest
W module jest sprawdzane, czy w pamięci są już wartości z koordynatów [U,V], [U+1,V] [U,V+1], [U+1,V+1]. 
Jeżeli któregoś piksela brakuje, jest on pobierany z pamięci i zapisywany w rejestrze przechowującym niepotrzebny piksel. 
Jeżeli wszystkie koordynaty się zgadzają , obliczana jest wartość piksela wyjściowego zgodnie ze wzorem \eqref{equ:bilinear}.  

\begin{equation}\label{equ:bilinear}
Ir = A(1-U_f)(1-V_f)+BU_f(1-V_f)+C(1-U_f)V_f+ D U_fV_f
\end{equation}
\noindent gdzie: $ A, B, C ,D $ odpowiadają wartościom pikseli w [U,V], [U+1,V] [U,V+1], [U+1,V+1], a $ Ir $ to wartość wyjściowa piksela wyjściowego. $U_f$ i $V_f$ stanowią U\_fraction i V\_fraction. %TODO no to może nie wprowadzać dwóch oznaczeń (w sensie _f i _fraction, tylko ujednolicić.)

Moduł działa strumieniowo. 
W przypadku gdy jest wymagana aktualizacja rejestrów strumień jest wstrzymywany.
biera wartość 4 otaczających, podanych na wejściu punktu, pikseli z BRAM i na ich bazie jest wykonywana interpolacja. %TODO styl. + czegoś brakuje na poczatku
Moduł zapamiętuje 4 ostatnio użyte piksele które są na bieżąco aktualizowane wraz z zmianą położenia punktu wejściowego na obrazie IR.
%TODO To juz było.... 


\subsection{Łączenie strumieni}

Moduł posiada dwa wejścia dla obrazu. 
Jeden strumień jest głównym i do niego jest dołączany drugi strumień. %TODO dodać który jest który
Do synchronizacji została wykorzystana możliwość wstrzymania transmisji poprzez AXI4-Stream. 
Piksele z dołączanego strumienia są odrzucane do momentu pojawienia się sygnału SOF.
W momencie pojawienia się sygnału SOF w strumieniu głównym transmisja zostaje wznowiona, pod kontrolą strumienia wyjściowego. %TODO a) za dużo strumień, poza tym jest to jakieś niejasne.
Po przejściu całej ramki strumienie są ponownie synchronizowane.  

\subsection{Koloryzacja i nakładanie}

Strumień RGBIR zostaje połączony w jeden obraz. 
Obraz IR zostaje poddany koloryzacji na podstawie 12-bitowego LUT i nałożony w proporcjach 50 na 50 z obrazem RGB. 
Na wyjściu jest podany 24 bitowy strumień RGB.

%TODO na pewno przykład obrazu, rozwinięcie LUT, jakieś szczegółby tej koloryzacji

\subsection{Obramowanie wyników}

Moduł dodaje do obrazu podanego na strumień wejściowy ramkę, która następnie jest podawana dalej strumieniem wyjściowym. %TODO chyba jest wyświetlana, a nie podawana
Parametry ramki są ustawiane przez dwa 32 bitowe rejestry. 
Pierwszy  (\texttt{position\_reg}) zawiera pozycję, gdzie ma się znajdować ramka na obrazie (lewy górny róg ramki), drugi (\texttt{parameters\_reg}) odpowiada za kolor i wielkość ramki. 
Rejestry są konfigurowane poprzez AXI4-Lite.

%TODO A kto je konfiguruje ?

\section{System procesorowy}

System procesorowy spełnia dwa podstawowe zadania: konfiguracja modułów zawartych w logice programowalnej za pomocą interfejsu AXI4-Lite, takich jak macierz projekcji, wartość progu binaryzacji i wartość progu mocy dopasowania dla modułu DPM, wzmocnienie oraz offset modułu normalizacji sygnału IR. %TODO ten offest to się pojawia po roaz pierwszy 
Pozwala on również na zapisanie na karcie SD aktualnej ramki bądź pozytywnie sklasyfikowanego obrazu okna detekcji, jak i obrazu do przeprowadzenia kalibracji.

Drugim zadaniem jest przeszukanie listy kandydatów w celu znalezienia tego z największą mocą dopasowania, wyliczenie cech HOG i klasyfikacji SVM. 
Oryginalny rozmiar okna detekcji w układzie kamery IR wynosi 16x40 zaś na obrazie RGBIR analizowane jest okno 80x192 piksele. 
Jest ono podzielone na 60 komórek o wielkości 16x16 pikseli. 
Następnie obliczane są gradienty oraz histogram dla każdej komórki. 
Wykorzystany jest histogram ważony o 9 przedziałach. %TODO prze to "ważony" rozumie Pan interpolację ?
Do każdego histogramu jest przypisana dodatkowo suma kwadratów wszystkich wartości przedziałów. %TODO ???
Następnie komórki są łączone w bloki 2 na 2, w obrębie których dokonuje się normalizacji wykorzystując wcześniej obliczone sumy kwadratów. 
Bloki nakładają się na siebie dając w sumie 44 bloki. %TODO Styl.
Suma histogramów z wszystkich bloków tworzy 1584 elementowy wektor cech. 
Wektor jest przemnożony przez wektor beta uzyskany w procesie nauczania SVM i dodany bias. %TODO tu że poddany klasyfiakcji (a to jak to proszę wzór na liniowy SVM podać w teorii - teraz nie ma, a tu ew. się na ten rozdział powoołać.)
Jeżeli uzyskany wynik jest większy od 0, badane okno zostaje sklasyfikowane z wynikiem pozytywnym.

%TODO Tu jeszce trzeba napisać w czym C/C++ ?, że to Pan pewnie sam napisał i tą kwestię komunikacji omówić szerzej (przerwania). Osobny akapit (co najmniej na SD)

%TODO Jakieś testy tego rozwiązania. Przynajmniej, że to działa - to jest podsawtawa. Zdjęcie working system (potem filmik - koniecznie).
%TODO Druga sprawa to ewaluacja samego podejścia na jakieś większej bazie - czy to jest możliwe ?

