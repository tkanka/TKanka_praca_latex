\chapter{Podsumowanie i dalsze kierunku rozwoju}
Zgodnie z celem pracy został opracowany system detekcji przechodniów na podstawie obrazu z kamery termowizyjnej. Mając do dyspozycji kamerę Lepton o bardzo małej rozdzielczości (80x60px) zdecydowano się na wykorzystanie obrazu wizyjnego i detekcję obiektów w przestrzeni multispektralnej (RGBIR). W tym celu, zgodnie z założeniem, wykorzystano heterogenicznego układu Zynq-7000 umożliwiającego sprzętowo-programową implementację algorytmów. Zadaniem części logiki programowalnej (FPGA) było połączenie strumieni wizyjnych z kamer w jeden obraz multispektralny oraz określenie obszaru zainteresowania (ROI) do klasyfikacji. W celu określenia ROI został wykorzystany słabszy klasyfikator: wzorzec probabilistyczny, który został opracowany przez autora w pracy inżynierskiej. Wytypowany kandydat był następnie klasyfikowany przy użyciu deskryptora HOG i SVM. Proces ten odbywał się w systemie procesorowym układu Zynq. Drugim zadaniem systemu procesorowego była konfiguracja parametrów modułów zaimplementowanych w logice programowalnej oraz prezentację wyników. Dodatkowa funkcja zapisania obrazów na karcie SD dała możliwość stworzenia własnej bazy próbek do nauczenia klasyfikatora. 
Kamera termowizyjna pracuję z prędkością 9 klatek na sekundę i jest źródłem obrazu dla modułu odpowiedzialnego za określanie ROI. Z racji że system procesorowy potrzebuje 75ms na klasyfikację pojedynczego ROI a kolejne ramki obrazu pojawiają się co około 111ms wykonywana jest tylko jedna detekcja na klatkę. Wynik ten jest znacznie słabszy od tych uzyskanych w podobnych rozwiązaniach.  W celu poprawy wyniku można by przenieść część obliczeń związaną z obliczaniem wektora cech z systemu procesorowego do logiki programowalnej ale nie pozwalają na to ograniczone zasoby układu FPGA.
