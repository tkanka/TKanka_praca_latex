\chapter{Podsumowanie i możliwe dalsze kierunki rozwoju systemu}

Zgodnie z celem pracy został opracowany system detekcji przechodniów na podstawie obrazu z kamery termowizyjnej. 
Mając do dyspozycji kamerę Lepton o~bardzo małej rozdzielczości (80x60 pikseli) zdecydowano się na wykorzystanie obrazu wizyjnego i detekcję obiektów w przestrzeni multispektralnej (RGBIR). 
W~tym celu, zgodnie z założeniem, wykorzystano heterogeniczny układ Zynq-7000 umożliwiający sprzętowo-programową implementację algorytmów.
Zadaniem części logiki programowalnej (FPGA) było połączenie strumieni wizyjnych z~kamer w~jeden obraz multispektralny oraz określenie obszaru zainteresowania (ROI) do klasyfikacji. 
W celu określenia ROI został wykorzystany słabszy klasyfikator: wzorzec probabilistyczny -- opracowany przez autora w~ramach pracy inżynierskiej. 
Wytypowany kandydat był następnie klasyfikowany przy użyciu deskryptora HOG i~SVM. 
Proces ten odbywał się w~systemie procesorowym układu Zynq. 
Drugim zadaniem systemu procesorowego była konfiguracja parametrów modułów zaimplementowanych w~logice programowalnej oraz wizualizacja wyników. 
Dodatkowa funkcja zapisania obrazów na karcie SD dała możliwość stworzenia własnej bazy próbek do nauczenia klasyfikatora.

Zastosowana kamera termowizyjna umożliwia akwizycję 9 klatek na sekundę i~jest źródłem obrazu dla modułu odpowiedzialnego za określanie ROI. 
Ponieważ system procesorowy potrzebuje 75 ms na klasyfikację pojedynczego ROI, a~kolejne ramki obrazu pojawiają się co około 111 ms możliwa jest tylko jedna detekcja na klatkę obrazu IR. 
Nie jest to imponujący wynik. %OK TODO słabszy to złe słowo, poza tym... co to są inne.

W~celu poprawy szybkości działania można by przenieść obliczanie deskryptora HOG związaną z~systemu procesorowego do logiki programowalnej, ale nie pozwalają na to ograniczone zasoby logiczne używanego układu Zynq. 
Na podstawie modelu programowego można stwierdzić, że w porównaniu do poprzedniego systemu spadła liczba fałszywych pozytywnych detekcji. 

Dalszym kierunkiem rozwoju byłoby dodanie drugiej kamery termowizyjnej Lepton, co pozwoliłoby na otrzymanie obrazu stereoskopowego, by estymować odległość od obiektu. Mając tę informację można automatycznie dopasowywać wielkość wzorca odpowiadającą sylwetce przechodnia w danej odległości od kamery. Mała rozdzielczość czujnika również zachęca do analizy ruchu. Poprzednią ramkę obrazu można przechować bezpośrednio w pamięci BRAM układu FPGA i wykorzystać do strumieniowego obliczania lokalnego przepływu optycznego. Na jego podstawie system wspomagania kierowcy mógłby wykryć czy wykryty przechodzień może wtargnąć przed pojazd.
%TODO 2 1. Raczej, że fuzja danych czy coś.
%OK TODO 2. Proszę już coś wymyleć inna kamera, analiza ruchu, śledzenie, itp. No na pewno ma Pan jakieś pomysły.