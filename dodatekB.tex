\chapter{Dodatek B -- Instrukcja obsługi systemu}

%TODO A tu nie trzeba jeszcze włożyć karty SD ? Bo rozumiem, że to tak startuje
%TODO Jeszcz taki informtyczny opis projektu - co jest w jakim pliku.

Po podpięciu platformy Zybo przez złącze mikro USB uzyskujemy dostęp do terminalu poprzez UART. 
Parametry portu: baud rate 115200, 8 bitów, jeden bit stopu, brak kontroli parzystości.

Wciśnięcie klawisza 'h' powoduje pokazania się karty pomocy:

\begin{lstlisting}
HELP:
1 VDMA status
2 Start detection
3 Restart DMA

5 read T matrix
6 write T matrix
7 write bin threshold
8 update LP threshold

w save next detected ROI
s save centered ROI
x save center ROI and HOG discriptor
d save whole frame
m mount SD card
q get calibration image

\end{lstlisting}
 
\noindent Wprowadzenie znaku z~powyższej listy do konsoli powoduje wykonanie następującej akcji:
 
\begin{itemize}
\item 1 -- wyświetlenie statusu VDMA,
\item 2 -- rozpoczęcie detekcji przechodniów,
\item 3 -- restart DMA,
\item 5 -- wyświetlenie parametrów macierzy transformacji projekcyjnej,
\item 6 -- pozwala na wprowadzenie nowej macierzy T. Dane należy wprowadzić w postaci 9 liczb z przecinkiem (kropką) np.: wprowadzenie ''1.0 0.0 0.0 0.0 1.0 0.0 0.0 0.0 1.0'' zamienia macierz T na macierz jednostkową,
\item 7 -- pozwala na wprowadzenie progu binaryzacji dla modułu DPM,
\item 8 -- pozwala na wprowadzenie progu LP dla modułu DPM,
\item w -- zapisanie następnego wykrytego ROI w~postaci pliku ROInnn.CIR gdze nnn to kolejne numery plików.
\item s -- zapisanie ROI znajdującego się na środku obrazu w~postaci pliku ROInnn.CIR gdzie nnn to kolejne numery plików.
\item x -- zapisanie ROI znajdującego się na środku obrazu w~postaci pliku ROInnn.CIR  oraz wektora cech w postaci pliku FVnnn.FLO nnn to kolejne numery plików.
\item d -- zapisanie całej ramki obrazu w~postaci pliku FRAMEnnn.CIR gdzie nnn to kolejne numery plików.
\item m -- zamontowanie karty SD w celu zapisu plików,
\item q -- zapis obrazu kalibracyjnego w~postaci pliku CALIBR.CIR. Spowoduje to zapisanie obrazu wizyjnego wraz z~niepoddanym transformacji obrazem termowizyjnym.
\end{itemize}

Pliki .CIR można otworzyć za pomocą funkcji [IrImage, rgbImage] = cir2image( filename ) w pakiecie MatLab (znjdującym się na płycie CD w folderze /src). Zwraca ona dwie macierze: IrImage -- zapisany obraz w podczerwieni oraz rgbImage zapisany obraz wizyjny.

Pliki .FLO składa się z 1584 32-bitowych liczb zmiennoprzecinkowych typu float. 
Reprezentuje on obliczony przez procesor wektor cech związanego z nim ROI.
