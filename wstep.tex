\chapter{Wprowadzenie}

Cyfrowa analiza obrazów znalazła szerokie zastosowanie w~wielu dziedzinach życia.
Dzięki niej możliwe jest automatyczne uzyskanie istotnych dla użytkownika informacji.
Przez ostanie kilkadziesiąt lat opracowano tysiące różnych technik i~algorytmów wyspecjalizowanych do określonych zadań np. leśne fotopułapki do badania zachowań i migracji zwierząt, systemu kontroli jakości i przebiegu procesu przemysłowego, metody kontroli dostępu poprzez rozpoznawanie twarzy m.in. w smartfonach, algorytmu analizy zdjęć satelitarnych ziemi umożliwiające prognozowanie pogody, sterowanie ruchem drogowym na podstawie obrazu z kamer zamontowanych nad skrzyżowaniami, a~także systemy do badania przekroju żołędzia pozwalające określić czy jest on dobrym kandydatem na sadzonkę.

Wzrok ludzki umożliwia rejestrację obrazu w~pewnym zakresie promieniowania elektromagnetycznego zwanego światłem widzialnym.
Dzisiejsza technologia daje możliwość rejestracji poza tym widmem.
Przykładem są kamery termowizyjne, które stają się coraz tańsze i przez to bardziej popularne.
Dostarczają one informację o temperaturze obserwowanych obiektów.
Jest to coraz chętniej wykorzystywane np. w weterynarii do określenia miejsc urazów zwierząt, w~przemyśle do kontroli jakości artykułów spożywczych, w~budownictwie do analizy strat cieplnych w budynkach, w~systemach wspomagania kierowcy do detekcji obiektów w~pobliży drogi, przez ratowników do odnajdywania zasypanych ludzi w~gruzowiskach, straż graniczną do monitorowania granic, przez wojsko do odnajdywania celów i~zagrożeń podczas misji m.in. z wykorzystaniem dronów \cite{gade2014thermal}.
%TODO 2 A tak sobie myślę, czy nie do urazów u ludzi ... 

Większość systemów wizyjnych służących do detekcji przechodniów jest oparta o~analizę obrazów z~zakresu światła widzialnego bądź podczerwieni.
W przypadku światła widzialnego można uzyskać bardzo dobre wyniki pod warunkiem, że wyszukiwane obiekty są dobrze oświetlone i~wyróżniają się swoim kolorem od tła.
Podczerwień, a~szczególnie termowizja, umożliwia detekcję w~warunkach nocnych i ograniczonej widoczności.
Oba podejścia mają swoje wady i~zalety, które wzajemnie się uzupełniają (np. duże nasłonecznienia powoduje, że tło termiczne staje się dużo wyższe, co utrudnia wyodrębnienie pieszego, natomiast potencjalnie daje idealne warunki do uzyskania dobrej jakości obrazu w~zakresie widzialnym) \cite{lee2015robust}.
Połączenie tych dwóch obrazów daje możliwość uzyskania jeszcze lepszej skuteczności rozpoznawania ludzi.
W pracy \cite{st2007combination} autorzy nazywają ten rozszerzony format jako RGBT (,,Red-Green-Blue-Thermal''), natomiast inna praca jako analizę multispektralną (ang. \textit{Multispectral}) \cite{hwang2015multispectral}, albo po prostu jako połączony obraz z kamery termowizyjnej i~wizyjnej \cite{lee2015robust}.

Skuteczna detekcja obiektów często wymaga dużego zapotrzebowania na zasoby obliczeniowe.
W wielu przypadkach nie da się uzyskać satysfakcjonującej wydajności -- tak by można było uznać system za działający w~czasie rzeczywistym -- wykorzystując jedynie typowy komputer wyposażony w~procesor ogólnego przeznaczenia (nawet najnowszej generacji).
Stosuje się zatem różne metody akceleracji obliczeń.
Jednym z~podejść jest zastosowanie kart graficznych (GPU ang. \textit{Graphics Processing Unit}).
Pozwalają one na duże zrównoleglenie obliczeń, jednak wciąż charakteryzują się znacznym zużyciem energii (choć należy zaznaczyć, że obecnie trwają intensywne prace nad poprawą ich efektywności energetycznej). 
Tworzenie specjalizowanych układów scalonych (ASIC ang. \textit{Application-Specific Integrated Circuit}) daje najlepsze rezultaty w~implementacji systemu wizyjnego, ale ich opracowanie i produkcja wymaga bardzo dużych nakładów finansowych. 

Dobre rozwiązanie stanowią układy rekonfigurowalne, które charakteryzują się podobnymi możliwościami w~realizacji wyspecjalizowanych zadań co układy ASIC, ale nie wymagają tworzenia całkiem nowych układów scalonych. 
Układy FPGA (ang. \textit{Field-Programmable Gate Array}) umożliwiają zrównoleglenie obliczeń i~są szeroko stosowane w systemach wizyjnych.
Szczególnie chętnie są wykorzystywane do realizacji operacji niskiego poziomu, przygotowując wstępnie obraz do dalszej analizy na wysokim poziomie.
Przykłady takich operacji to: filtry konwolucyjne, filtry 2D, podpróbkowanie, wykrywanie krawędzi, obliczanie SAD (ang. \textit{Sum Of Absolute Differences}) z~regionu zainteresowania, obliczanie orientacji krawędzi i~histogramów, obliczanie strumieniowo statystyk (wartość maksymalna, minimalna, średnia), zmiana przestrzeni barw \cite{kisacanin2008embedded}. 
Dodatkową zaletą układów FPGA jest mały pobór mocy, co czyni je niezwykle atrakcyjne dla aplikacji mobilnych -- takich jak drony czy czujniki środowiskowe~\cite{garcia2014survey}.

Układy heterogeniczne łączą w~jednej obudowie dwa układy (zasoby obliczeniowe) o~różanej architekturze i funkcjonalności.
Przykładem takiego połączenia jest Zynq-7000 firmy Xilinx, który integruje w~sobie układ FPGA oraz procesor ARM.
Największą zaletą takiego rozwiązania jest wysoka przepustowość transferu danych między procesorem, a~logiką programowalną.
%TODO 2 Tu może jeszcze coś o możliwości podzialu algorymu HW/SW

Niniejsza praca stanowi kontynuację i rozwinięcie pracy inżynierskiej autora.

\section{Cel pracy}

Celem pracy była realizacja wbudowanego systemu wizyjnego do detekcji wybranych obiektów (np. ludzi) na podstawie obrazu z kamery termowizyjnej. 
Założono, że jako platforma obliczeniowa zostanie użyty układ heterogeniczny (np. Zynq firmy Xilinx), który umożliwia implementację sprzętowo-programową algorytmów.
W~pierwszym etapie pracy należało dokonać przeglądu i~oceny zrealizowanego w~ramach pracy inżynierskiej algorytmu, a~także przeprowadzić pogłębioną analizę literatury związanej z~tematem. 
Należało przy tym zwrócić szczególną uwagę na systemy detekcji pieszych (wspomaganie kierowcy) oraz systemy dla autonomicznych pojazdów latających (dronów). 
Ponadto należało przeanalizować możliwość wykorzystania kontekstu czasowego tj. informacji zawartej w~sekwencji obrazów. 
Na tej podstawie należało wytypować algorytmy, które zostaną zaimplementowane i przetestowane w aplikacji programowej (Matlab/C++/Python/OpenCV). 
Należało również zgromadzić bazę zdjęć lub sekwencji testowych.

W drugim etapie należało wybrane wspólnie z opiekunem pracy algorytmy zaimplementować w systemie programowo sprzętowym, uruchomić oraz sprawdzić ich skuteczność w różnych scenariuszach testowych.

Oczekiwanym rezultatem pracy był: opis wykorzystania informacji termowizyjnej w~detekcji pieszych (systemy wspomagania kierowcy) oraz detekcji ludzi z wykorzystaniem dronów, implementacja i~analiza kilku podejść (w aplikacji programowej), implementacja sprzętowo-programowo wybranych algorytmów detekcji.

\section{Struktura pracy}

Pracę rozpoczyna wstęp, w którym przestawiono możliwości i~zastosowania kamer termowizyjnych. 
Wskazano również zalety wykorzystania podczerwieni do rozszerzenia spektrum konwencjonalnej kamery wizyjnej -- analizy multispektralnej.
W~rozdziale \ref{cha:multispectral} zostały omówione sposoby uzyskania obrazu multispektralnego. 
Rozdział rozpoczyna się od scharakteryzowania promieniowania podczerwonego oraz metody jego rejestracji. 
Następnie został zaprezentowany model kamery otworkowej oraz transformacja projekcyjna, które stanowią podstawę teoretyczną do prawidłowego połączenia obrazów z dwóch różnych kamer.
W~rozdziale \ref{cha:algoDetPiesz} przedstawiono ogólny przegląd algorytmów służących do detekcji pieszych. 
W~poszczególnych sekcjach zostały omówione poszczególne etapy przetwarzania obrazu i~klasyfikacji wraz z~ogólnie stosowanymi rozwiązaniami.%TODO 2 powt. poszczególnych
Następnie w~rozdziale \ref{cha:fpga} zostały omówione sposoby zrównoleglenia obliczeń i~możliwość FPGA na przykładzie układu heterogenicznego Zynq-7000 firmy Xilinx.
Kolejny rozdział \ref{cha:przegLiter} zawiera streszczenia kilku wybranych artykułów, które poruszają tematykę związaną z~niniejszą pracą.
W~rozdziale \ref{cha:propSysWiz} przedstawiono koncepcję zaproponowanego rozwiązania wraz ze szczegółami dotyczącymi jego wykonania.
Pracę zakończono podsumowaniem.

